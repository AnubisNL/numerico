\documentclass{article}

% Language setting

\usepackage[brazil]{babel}

% Set page size and margins
% Replace `letterpaper' with `a4paper' for UK/EU standard size
\usepackage[letterpaper,top=2cm,bottom=2cm,left=3cm,right=3cm,marginparwidth=1.75cm]{geometry}

% Useful packages
\usepackage{amsmath}
\usepackage{graphicx}
\usepackage[colorlinks=true, allcolors=blue]{hyperref}
\usepackage{tikz ,tkz-base, tkz-fct}
\usepackage{upquote}
\usepackage{listings}
\lstdefinelanguage{JavaScript}{
  morekeywords={typeof, new, true, false, catch, function, return, null, catch, switch, var, if, in, while, do, else, case, break},
  morecomment=[s]{/*}{*/},
  morecomment=[l]//,
  morestring=[b]",
  morestring=[b]'
}


\title{Lista de Fevereiro}
\author{Natan Ledur}

\begin{document}
\maketitle

\section{Exercício }
\begin{equation*}
      A= \left\{\begin{matrix} x + 2y = 2 \\  -3x + y=2 \end{matrix}\right. 
       \ \
      B = \left\{\begin{matrix} 5x - y = 10 \\ -\frac{1}{5} + \frac{y}{25} = -\frac{2}{5} \\ 3x+4y=2 \end{matrix}\right.
      \ \
      C = \left\{\begin{matrix}  x + y - z = 2 \\ -x -y +z = 0 \\ x+2y-3z=4 \end{matrix}\right.
      \end{equation*}
      \begin{itemize}
          \item Para o sistema A temos;
      \end{itemize}
      \begin{equation*}
      A=\begin{pmatrix}  1&2  &2 \\   -3&  1& 2\end{pmatrix}
      \xrightarrow[]{L_2=L_2 - (-3)L_1}
      \begin{pmatrix} 1 &2  &2 \\  0& 7 & 8 \end{pmatrix}\displaystyle \displaystyle 
      \end{equation*}
       
      \begin{equation}
      A = \left\{\begin{matrix} x+2y=2\\  7y=8 \end{matrix}\right.
      \end{equation}
      \\
      Resolvendo o sistema (1) temos,$\left [ \left [ y=\frac{8}{7}  , x=-\frac{2}{7}  \right ] \right ]$ , com isso vemos que o sistema A é consistente.
      \begin{center}
        \begin{tikzpicture}
            \tkzInit[   xmin=-3, xmax=3,
                        ymin=-3, ymax=3]
            \tkzDrawX
            \tkzDrawY        
             
            % primeira reta
            \draw[domain=-2:3,variable=\x,red] plot ({\x},{(2-\x)/2});
            % segunda reta
            \draw[domain=-1:0.5,variable=\x,green] plot ({\x},{3*\x+2});
  
            % ponto de interseção
            \filldraw[black] (-0.285714,1.1428) circle (0.05cm) node[anchor=west] {$(-2/7,8/7)$};
            
        \end{tikzpicture}  
        \end{center}


      % Sistema B
      \begin{itemize}
          \item Para o sistema B temos;\\               
          É possível notar que no sistema B a segunda equação é a primeira equação multiplicado por $ -\frac{1}{25}$, logo podemos analisar uma das duas com a ultima equação e teremos nosso resultado. 
          
      \end{itemize}
      \begin{equation*}
      \begin{pmatrix} 5 &-1  &10 \\-\frac{1}{5}& \frac{1}{25} & -\frac{2}{5} \\ 3 & 4 & 2 \end{pmatrix}
      \xrightarrow[]{L_3=L_3-(\frac{3}{5})L_1}
      \begin{pmatrix} 5 &-1  &10 \\-\frac{1}{5}& \frac{1}{25} & -\frac{2}{5} \\ 0 & \frac{23}{5} & -\frac{20}{5} \end{pmatrix}
      \end{equation*}
      Obtemos assim um sistema mais simples de se resolver:
      \begin{equation}
        B = \left\{\begin{matrix} 5x - y = 10 \\ \medskip -\frac{1}{5} + \frac{y}{25} = -\frac{2}{5} \\  \frac{23}{5}y= -\frac{20}{5} \end{matrix}\right.
      \end{equation}
       \\ 
       Resolvendo o sistema (2) temos que,$\left [ \left [ y=-\frac{20}{23}  , x=\frac{42}{23}  \right ] \right ]$ , com isso vemos que o sistema B é consistente.\bigskip
    \\ \\ Podemos analisar também através do gráfico das funções, que o ponto de intercessão é $ y=-\frac{20}{23}  , x=\frac{42}{23} $.
    
    \begin{center}
    \begin{tikzpicture}
        \tkzInit[   xmin=-3, xmax=3,
                    ymin=-3, ymax=3]
        \tkzDrawX
        \tkzDrawY        
         
        % primeira reta
        \draw[domain=1.3:2.5,variable=\x,red] plot ({\x},{5*\x-10/1});
        % segunda reta
        \draw[domain=-2:3,variable=\x,green] plot ({\x},{(2-3*\x)/4});
         % ponto de interseção
        \filldraw[black] (1.826,-0.869) circle (0.05cm) node[anchor=west] {$(\frac{42}{23},-\frac{20}{23})$};
        
    \end{tikzpicture}  
    \end{center}

      \begin{itemize}
          \item Para o sistema C temos;
      \end{itemize}
    \begin{equation*}\begin{pmatrix}
      1 & 2 & -1 &2 \\ 
       -1& -1 &  1& 0\\ 
       1&  2&  -3& 4
      \end{pmatrix}
      \xrightarrow[]{L_2=L_2 - (-1)L_1} 
      \begin{pmatrix}
      1 & 2 & -1 &2 \\ 
       0& 0 &  0& 2\\ 
       1&  2&  -3& 4
      \end{pmatrix}\end{equation*}
 Resultando na matriz
\begin{equation}
  C = \left\{\begin{matrix}  x + y - z = 2 \\ 0 = 2 \\ x+2y-3z=4 \end{matrix}\right.
\end{equation} 
\\
Como $0=2$ é um absurdo o sistema (3) não pode ser resolvido logo o sistema C não é consistente.\\
Uma forma mais rápida de ver que o sistema não é consistente e ver que a primeira e a segunda equação do sistema são planos paralelos, sendo assim não possuem interseção.

\section{Exercício }

\begin{lstlisting}[language=JavaScript, caption=métodos de eliminação de Gauss e Gauss com pivotamento.]
  
  function gausspivo(A, b) {
    const n = A.length;
    const Ab = new Array(n);
  
   
    for (let i = 0; i < n; i++) {
      Ab[i] = [...A[i], b[i]];
    }

    for (let i = 0; i < n; i++) {
    
      let max = Math.abs(Ab[i][i]);
      let maxIndex = i;
      for (let j = i + 1; j < n; j++) {
        if (Math.abs(Ab[j][i]) > max) {
          max = Math.abs(Ab[j][i]);
          maxIndex = j;
        }
      }
  
      
      if (maxIndex !== i) {
        [Ab[i], Ab[maxIndex]] = [Ab[maxIndex], Ab[i]];
      }
  
      
      for (let j = i + 1; j < n; j++) {
        const factor = Ab[j][i] / Ab[i][i];
        for (let k = i; k <= n; k++) {
          Ab[j][k] -= factor * Ab[i][k];
        }
      }
    }
  
    
    const x = new Array(n);
    for (let i = n - 1; i >= 0; i--) {
      let sum = 0;
      for (let j = i + 1; j < n; j++) {
        sum += Ab[i][j] * x[j];
      }
      x[i] = (Ab[i][n] - sum) / Ab[i][i];
    }
  
    return x;
  }
  function gauss(A, b) {
    const n = A.length;
    const Ab = new Array(n);
  
    
    for (let i = 0; i < n; i++) {
      Ab[i] = [...A[i], b[i]];
    }
  
    
    for (let i = 0; i < n - 1; i++) {
      for (let j = i + 1; j < n; j++) {
        const factor = Ab[j][i] / Ab[i][i];
        for (let k = i; k <= n; k++) {
          Ab[j][k] -= factor * Ab[i][k];
        }
      }
    }
  
    
    const x = new Array(n);
    for (let i = n - 1; i >= 0; i--) {
      let sum = 0;
      for (let j = i + 1; j < n; j++) {
        sum += Ab[i][j] * x[j];
      }
      x[i] = (Ab[i][n] - sum) / Ab[i][i];
    }
  
    return x;
  }
  function decomposeLU(A) {
    const n = A.length;
    const L = new Array(n).fill(null).map(() => new Array(n).fill(0));
    const U = new Array(n).fill(null).map(() => new Array(n).fill(0));
  
    // Inicializa L com a diagonal principal igual a 1
    for (let i = 0; i < n; i++) {
      L[i][i] = 1;
    }
  
    
    for (let i = 0; i < n; i++) {
      
      let maxRow = i;
      for (let j = i + 1; j < n; j++) {
        if (Math.abs(A[j][i]) > Math.abs(A[maxRow][i])) {
          maxRow = j;
        }
      }
  
      
      [A[i], A[maxRow]] = [A[maxRow], A[i]];
      [L[i], L[maxRow]] = [L[maxRow], L[i]];
  
      
      for (let j = i + 1; j < n; j++) {
        const factor = A[j][i] / A[i][i];
        L[j][i] = factor;
        for (let k = i; k < n; k++) {
          A[j][k] -= factor * A[i][k];
        }
      }
    }
  
    
    for (let i = 0; i < n; i++) {
      for (let j = i; j < n; j++) {
        U[i][j] = A[i][j];
      }
    }
  
    return { L, U };
  }
  
  const A = [[0.4096, 0.1234, 0.3678,0.2943],
             [0.2246, 0.3872, 0.4015,0.1129],
             [0.3645, 0.1920, 0.3781,0.0643],
             [0.1784,0.4002,0.2786,0.3927]]
  const b = [0.4043, 0.1550, 0.4240,0.2557]
  const x = gauss(A, b);
  console.log(x); // [ 2, 1, -1 ]
  const a = gausspivo(A, b);
  console.log(a); // [ 2, 1, -1 ]
  const { L, U } = decomposeLU(A);
  console.log("L = ", L);
  console.log("U = ", U);
  \end{lstlisting}
  \begin{lstlisting}[language=JavaScript, caption=Execussão do codido]
    node main.js
    [3.4605863893500874,
    1.56095288560023,  
    -2.934233975722392,
    -0.4300595137280605]
    
    [3.4605863893500883, 
    1.560952885600231,  
    -2.9342339757223943,
    -0.4300595137280602]


    L =  [[ 1, 0, 0, 0 ],
        [ 0.435546875, 0, 0, 1 ],
        [ 0.54833984375, 0.9223022681839759, 0, 0 ],
        [ 0.8898925781249999, 0.23722448222558723, 0.2506079359717326, 0 ]]

    U =  [[ 0.4096, 0.1234, 0.3678, 0.2943 ],
        [ 0, 0.346453515625, 0.118405859375, 0.26451855468749996 ],
        [ 0, 0, 0.09061461280091464, -0.2924424789806533 ],
        [ 0, 0, 0, -0.18705725686919206 ]]
  \end{lstlisting}
   

  




\end{document}


 

