\documentclass{article}

% Language setting

\usepackage[brazil]{babel}

% Set page size and margins
% Replace `letterpaper' with `a4paper' for UK/EU standard size
\usepackage[letterpaper,top=2cm,bottom=2cm,left=3cm,right=3cm,marginparwidth=1.75cm]{geometry}

% Useful packages
\usepackage{amsmath}
\usepackage{graphicx}
\usepackage[colorlinks=true, allcolors=blue]{hyperref}
\usepackage{tikz ,tkz-base, tkz-fct}
\title{Lista de Fevereiro}
\author{Natan Ledur}

\begin{document}
\maketitle

\section{Exercício 1.}
\begin{equation*}
      A= \left\{\begin{matrix} x + 2y = 2 \\  -3x + y=2 \end{matrix}\right. 
       \ \
      B = \left\{\begin{matrix} 5x - y = 10 \\ -\frac{1}{5} + \frac{y}{25} = -\frac{2}{5} \\ 3x+4y=2 \end{matrix}\right.
      \ \
      C = \left\{\begin{matrix}  x + y - z = 2 \\ -x -y +z = 0 \\ x+2y-3z=4 \end{matrix}\right.
      \end{equation*}
      \begin{itemize}
          \item Para o sistema A temos;
      \end{itemize}
      \begin{equation*}
      A=\begin{pmatrix}  1&2  &2 \\   -3&  1& 2\end{pmatrix}
      \xrightarrow[]{L_2=L_2 - (-3)L_1}
      \begin{pmatrix} 1 &2  &2 \\  0& 7 & 8 \end{pmatrix}\displaystyle \displaystyle 
      \end{equation*}
       
      \begin{equation}
      A = \left\{\begin{matrix} x+2y=2\\  7y=8 \end{matrix}\right.
      \end{equation}
      \\
      Resolvendo o sistema (1) temos,$\left [ \left [ y=\frac{8}{7}  , x=-\frac{2}{7}  \right ] \right ]$ , com isso vemos que o sistema A é consistente.
      \begin{center}
        \begin{tikzpicture}
            \tkzInit[   xmin=-3, xmax=3,
                        ymin=-3, ymax=3]
            \tkzDrawX
            \tkzDrawY        
             
            % primeira reta
            \draw[domain=-2:3,variable=\x,red] plot ({\x},{(2-\x)/2});
            % segunda reta
            \draw[domain=-1:0.5,variable=\x,green] plot ({\x},{3*\x+2});
  
            % ponto de interseção
            \filldraw[black] (-0.285714,1.1428) circle (0.05cm) node[anchor=west] {$(-2/7,8/7)$};
            
        \end{tikzpicture}  
        \end{center}


      % Sistema B
      \begin{itemize}
          \item Para o sistema A temos;
          \\          
          É possível notar que no sistema B a segunda equação é a primeira equação multiplicado por $ -\frac{1}{25}$, logo podemos podemos analisar uma das duas com a ultima equação e teremos nosso resultado. 
          
      \end{itemize}
      \begin{equation*}
      \begin{pmatrix} 5 &-1  &10 \\-\frac{1}{5}& \frac{1}{25} & -\frac{2}{5} \\ 3 & 4 & 2 \end{pmatrix}
      \xrightarrow[]{L_3=L_3-(\frac{3}{5})L_1}
      \begin{pmatrix} 5 &-1  &10 \\-\frac{1}{5}& \frac{1}{25} & -\frac{2}{5} \\ 0 & \frac{23}{5} & -\frac{20}{5} \end{pmatrix}
      \end{equation*}
      Obtemos assim um sistema mais simples de se resolver:
      \begin{equation}
        B = \left\{\begin{matrix} 5x - y = 10 \\ \medskip -\frac{1}{5} + \frac{y}{25} = -\frac{2}{5} \\  \frac{23}{5}y= -\frac{20}{5} \end{matrix}\right.
      \end{equation}
       \\ 
       Resolvendo o sistema (2) temos que,$\left [ \left [ y=-\frac{20}{23}  , x=\frac{42}{23}  \right ] \right ]$ , com isso vemos que o sistema B é consistente. \\
    \\ Podemos analisar também através do gráfico das funções, onde o ponto de intercessão é $ y=-\frac{20}{23}  , x=\frac{42}{23} $
    
    \begin{center}
    \begin{tikzpicture}
        \tkzInit[   xmin=-3, xmax=3,
                    ymin=-3, ymax=3]
        \tkzDrawX
        \tkzDrawY        
         
        % primeira reta
        \draw[domain=1.3:2.5,variable=\x,red] plot ({\x},{5*\x-10/1});
        % segunda reta
        \draw[domain=-2:3,variable=\x,green] plot ({\x},{(2-3*\x)/4});
         % ponto de interseção
        \filldraw[black] (1.826,-0.869) circle (0.05cm) node[anchor=west] {$(\frac{42}{23},-\frac{20}{23})$};
        
    \end{tikzpicture}  
    \end{center}

      \begin{itemize}
          \item Para o sistema C temos;
      \end{itemize}
      

\end{document}
    

